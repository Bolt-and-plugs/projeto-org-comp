%%%%%%%%%%%%%%%%%%%%%%%%%%%%%%%%%%%%%%%%%%%%%%%%%%%%%%%%%%%%%%%%%%%%%%%%%%%%
%%%%%%%%%%%%%%%%%%%%%%%%%%%%%%%%%%%%%%%%%%%%%%%%%%%%%%%%%%%%%%%%%%%%%%%%%%%%
% Article
%%%%%%%%%%%%%%%%%%%%%%%%%%%%%%%%%%%%%%%%%%%%%%%%%%%%%%%%%%%%%%%%%%%%%%%%%%%%
%%%%%%%%%%%%%%%%%%%%%%%%%%%%%%%%%%%%%%%%%%%%%%%%%%%%%%%%%%%%%%%%%%%%%%%%%%%%
% Unesp
%%%%%%%%%%%%%%%%%%%%%%%%%%%%%%%%%%%%%%%%%%%%%%%%%%%%%%%%%%%%%%%%%%%%%%%%%%%%
%%%%%%%%%%%%%%%%%%%%%%%%%%%%%%%%%%%%%%%%%%%%%%%%%%%%%%%%%%%%%%%%%%%%%%%%%%%%
\documentclass[a4paper,times,12pt]{article}
\usepackage{amsthm}
\usepackage[figuresright]{rotating}
\usepackage{graphicx}
\usepackage{booktabs}
\graphicspath{ {./images/} }
\usepackage{amssymb}
\usepackage{graphicx}
\usepackage{fancybox}
\usepackage{amsmath}
\usepackage{picinpar}
\usepackage{colortbl}
\usepackage{wasysym}
\usepackage{txfonts}
\usepackage{pb-diagram}
\usepackage{relsize}
\usepackage{tikz}
\usepackage{pgfplots}
\usepackage{subfigure}
\usepackage{algorithm}
\usepackage{algorithmic}
\usepackage{geometry}

\geometry{
  a4paper,         % Tamanho do papel
  left=2.5cm,      % Margem esquerda
  right=2.5cm,     % Margem direita
  top=2.5cm,         % Margem superior
  bottom=2.5cm       % Margem inferior
}

\begin{document}
\title{Apple M-Series}
\author{Carlos Eduardo Nogueira Silva\\Felipe Gomes da Silva \\Renan Sinhorini Pimentel\\ \\Instituto de Bioci\^{e}ncias, Letras e Ci\^{e}ncias Exatas, Unesp - \\ Univ Estadual Paulista (S\~{a}o Paulo State University) Rua Crist\'{o}v\~{a}o \\ Colombo 2265, Jd Nazareth, 15054-000, S\~{a}o Jos\'{e} do Rio Preto - SP, \\ Brazil.}
\maketitle
%%%%%%%%%%%%%%%%%%%%%%%%%%%%%%%%%%%%%%%%%%%%%%%%%%%%%%%%%%%%%%%%%%%%%%%%%%%%%
%%%%%%%%%%%%%%%%%%%%%%%%%%%%%%%%%%%%%%%%%%%%%%%%%%%%%%%%%%%%%%%%%%%%%%%%%%%%%

\section{Introdução}
\hspace*{+15pt} 

\section{História}
\hspace*{+15pt} 
A série M de chips da Apple representa uma mudança significativa na arquitetura dos computadores Mac, iniciada com o M1 em 2020. Não obstante, série M também representa a excelência em todas a área comercial de arquitetura de computadores. Esse processador, baseado em ARM, foi projetado para integrar CPU, GPU e memória unificada, permitindo eficiência energética e desempenho sem precedentes. Em 2021, a Apple lançou os M1 Pro e M1 Max, aprimorando essas capacidades com maior largura de banda de memória, mais núcleos de processamento e melhorias em gráficos, voltados para profissionais que exigem alto desempenho em aplicações gráficas e científicas.

De acordo com estudos na IEEE, o M1 é um exemplo de SoC (System on a Chip), integrando diversas funcionalidades em um único chip, aumentando a eficiência energética. Isso é possível graças à arquitetura ARM, que reduz o consumo em relação aos processadores tradicionais de desktop.

Esse avanço é uma parte essencial da estratégia da Apple para ter controle total sobre o hardware e o software de seus dispositivos, otimizando o desempenho para suas necessidades específicas, como maior duração da bateria e melhor desempenho em tarefas gráficas e de IA.

\section{Arquitetura e Organização}
\hspace{+15pt}

\section{Desempenho}
\hspace{+15pt}


\section{Comparativo entre modelos}
\hspace{+15pt}


\section{Conclusão}
\hspace{+15pt}



\newpage
\begin{thebibliography}{00}
\bibitem{brain_like_computing} Ou, W., Xiao, S., Zhu, C., Han, W., \& Zhang, Q. (2022). 
\textbf{An overview of brain-like computing: Architecture, applications, and future trends}. Frontiers in Neurorobotics, 16, 1041108.
\bibitem{quantum_basics}Kane, B. (1998). 
\textbf{A silicon-based nuclear spin quantum computer. } Nature, 393, 133-137.
\bibitem{quantum_bits} Raj, G., Singh, D., e Madaan, A. (2018).
\textbf{ Analysis of Classical and Quantum Computing Based on Grover and Shor Algorithm. }, 413-423.
\end{thebibliography}
\end{document}
