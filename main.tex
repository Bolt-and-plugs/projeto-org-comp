%%%%%%%%%%%%%%%%%%%%%%%%%%%%%%%%%%%%%%%%%%%%%%%%%%%%%%%%%%%%%%%%%%%%%%%%%%%%
%%%%%%%%%%%%%%%%%%%%%%%%%%%%%%%%%%%%%%%%%%%%%%%%%%%%%%%%%%%%%%%%%%%%%%%%%%%%
% Article
%%%%%%%%%%%%%%%%%%%%%%%%%%%%%%%%%%%%%%%%%%%%%%%%%%%%%%%%%%%%%%%%%%%%%%%%%%%%
%%%%%%%%%%%%%%%%%%%%%%%%%%%%%%%%%%%%%%%%%%%%%%%%%%%%%%%%%%%%%%%%%%%%%%%%%%%%
% Unesp
%%%%%%%%%%%%%%%%%%%%%%%%%%%%%%%%%%%%%%%%%%%%%%%%%%%%%%%%%%%%%%%%%%%%%%%%%%%%
%%%%%%%%%%%%%%%%%%%%%%%%%%%%%%%%%%%%%%%%%%%%%%%%%%%%%%%%%%%%%%%%%%%%%%%%%%%%
\documentclass[a4paper,times,12pt]{article}
\usepackage{amsthm}
\usepackage[figuresright]{rotating}
\usepackage{graphicx}
\usepackage{booktabs}
\graphicspath{ {./images/} }
\usepackage{amssymb}
\usepackage{graphicx}
\usepackage{fancybox}
\usepackage{amsmath}
\usepackage{picinpar}
\usepackage{colortbl}
\usepackage{wasysym}
\usepackage{txfonts}
\usepackage{pb-diagram}
\usepackage{relsize}
\usepackage{tikz}
\usepackage{pgfplots}
\usepackage{subfigure}
\usepackage{algorithm}
\usepackage{algorithmic}
\usepackage{geometry}

\geometry{
  a4paper,         % Tamanho do papel
  left=2.5cm,      % Margem esquerda
  right=2.5cm,     % Margem direita
  top=2.5cm,         % Margem superior
  bottom=2.5cm       % Margem inferior
}

\begin{document}
\title{Apple M-Series}
\author{Carlos Eduardo Nogueira Silva\\Felipe Gomes da Silva \\Renan Sinhorini Pimentel\\ \\Instituto de Bioci\^{e}ncias, Letras e Ci\^{e}ncias Exatas, Unesp - \\ Univ Estadual Paulista (S\~{a}o Paulo State University) Rua Crist\'{o}v\~{a}o \\ Colombo 2265, Jd Nazareth, 15054-000, S\~{a}o Jos\'{e} do Rio Preto - SP, \\ Brazil.}
\maketitle
%%%%%%%%%%%%%%%%%%%%%%%%%%%%%%%%%%%%%%%%%%%%%%%%%%%%%%%%%%%%%%%%%%%%%%%%%%%%%
%%%%%%%%%%%%%%%%%%%%%%%%%%%%%%%%%%%%%%%%%%%%%%%%%%%%%%%%%%%%%%%%%%%%%%%%%%%%%

\section{Introdução}
\hspace*{+15pt} 
A série de processadores Apple M marca uma mudança significativa na estratégia no ambiente corporativo do desenvolvimento de chips, culminando em uma arquitetura que alia alta eficiência energética com desempenho robusto. A transição da Apple entre processadores x86-64 para essa nova série começou em 2020, quando decidiu-se abandonar a Intel e passaram a utilizar-se de chips próprios em seus produtos, trazendo o mesmo nível de controle e otimização que já havia conquistado com os processadores da série A em dispositivos móveis.

Vale ressaltar que não foi encontrado informações de quando o desenvolvimento desta tecnologia iniciou-se, mas é claro que demandou muito tempo e embasamento em várias outras conquistas da área de arquitetura e organização de computadores. Neste artigo, iremos analizar tanto a história, o desenvolvimento, a arquitetura e organização e o tão esperado desempenho destes inovadores chips, tendo em vista promover uma visão, ora geral, ora profunda, da arquitetura ARM em aparelhos comerciais.

\section{História}
\hspace{+15pt} 
A série M de chips da Apple representa uma mudança significativa na arquitetura dos computadores Mac, iniciada com o M1 em 2020. Não obstante, esta série de processadores também representa a excelência em toda a área comercial de arquitetura de computadores. Esses processadores, baseado em ARM, foi projetado para integrar CPU, GPU e memória unificada, permitindo eficiência energética e desempenho sem precedentes. Em 2021, a Apple lançou os M1 Pro e M1 Max, aprimorando essas capacidades com maior largura de banda de memória, mais núcleos de processamento e melhorias em gráficos, voltados para profissionais que exigem alto desempenho em aplicações gráficas e científicas.
\subsection{Primeira Geração: Apple M1}
\hspace{+15pt} 
O primeiro processador da série, o M1, foi lançado em novembro de 2020. Esse SoC (System on a Chip) foi baseado na arquitetura ARM, tradicionalmente usada em dispositivos móveis, e incluía uma CPU de oito núcleos e uma GPU integrada com até oito núcleos. O M1 se destacou por sua eficiência energética, ao mesmo tempo que oferecia um desempenho impressionante, permitindo à Apple melhorar a autonomia de seus laptops e reduzir a dissipação de calor. Com o M1, a Apple também introduziu uma arquitetura de memória unificada, na qual CPU, GPU e outros componentes compartilham a mesma memória, o que reduz a latência e aumenta a eficiência.
Em 2021, a Apple expandiu a linha M1 com os modelos M1 Pro e M1 Max, que ofereceram melhorias substanciais em termos de capacidade de processamento e memória. Esses chips, voltados para profissionais, trouxeram GPUs mais poderosas, suportando tarefas complexas de edição de vídeo e renderização gráfica. Em 2022, a Apple revelou o M1 Ultra, combinando dois M1 Max em um único chip, usando uma interconexão chamada UltraFusion, com desempenho similar ao de estações de trabalho, reforçando o compromisso da Apple com a escalabilidade e a modularidade de sua arquitetura própria.

\subsection{Segunda Geração: Apple M2}
\hspace{+15pt} 
Em junho de 2022, a Apple apresentou o M2, uma atualização que oferecia desempenho até 18\% superior ao M1, com uma GPU aprimorada e um aumento na largura de banda da memória. A série M2 continuou a tradição de alta eficiência energética, enquanto introduzia avanços em aprendizado de máquina e IA. A Apple também lançou variantes do M2, como o M2 Pro e o M2 Max, que oferecem desempenho gráfico superior e são projetados para aplicativos que exigem processamento gráfico intenso
\subsection{Terceira Geração: Apple M3}
\hspace{+15pt}
Lançado em outubro de 2023, o M3 é o primeiro processador da série M fabricado com o processo de 3 nm, trazendo melhorias significativas em desempenho e eficiência energética. Este chip é capaz de executar tarefas complexas, como ray tracing em tempo real, graças ao aumento de núcleos tanto na CPU quanto na GPU. A Apple também lançou versões avançadas do M3, o M3 Pro e o M3 Max, que incluem até 16 núcleos de CPU e 40 núcleos de GPU, com capacidades de memória RAM que chegam a 128 GB, tornando-os ideais para cargas de trabalho avançadas e uso profissional.

Esse avanço é uma parte essencial da estratégia da Apple para ter controle total sobre o hardware e o software de seus dispositivos, otimizando o desempenho para suas necessidades específicas, como maior duração da bateria e melhor desempenho em tarefas gráficas e de IA. 

Vale a consideração de que estes processadores só puderam ser desenvolvidos devido devido à arquitetura ARM e o Conjunto Reduzido de Instruções (RISC), tópicos que abordaremos no próximo capítulo.

\section{Arquitetura e Organização}

\hspace*{+15pt} A série de chips M da Apple representa uma das arquiteturas de System-on-a-Chip (SoC) mais avançadas no mercado. Esses chips integram múltiplos componentes, como CPU, GPU, memória e unidades de processamento neural, em um único chip de silício, permitindo maior eficiência energética e uma integração mais profunda entre os componentes do hardware e o software.

Os chips da série M possuem um grande número de transistores, com o M1 contando com 16 bilhões, sendo capaz de fazer até 11 trilhões de operações por segundo. Além disso, a GPU integrada nesses chips conta com 8 núcleos no M1 e 32 núcleos no M1 Max, permitindo um processamento gráfico robusto sem a necessidade de uma GPU dedicada. Mas atualmente, com o chip M3, a Apple deu mais um salto significativo em termos de potência e eficiência, trazendo uma quantidade ainda maior de transistores, ultrapassando os 25 bilhões, o que permite uma capacidade de processamento ainda mais avançada e otimizada para tarefas intensivas de cálculo. Além disso, a GPU integrada no M3 também foi aprimorada, contando com até 40 núcleos em sua versão mais poderosa, oferecendo desempenho gráfico excepcional.

\begin{figure}[h]
    \centering
    \includegraphics[width=0.8\textwidth]{https://upload.wikimedia.org/wikipedia/commons/thumb/8/83/Apple_M1.jpg/1200px-Apple_M1.jpg}
    \caption{Chip M1}
    \label{fig:apple_m1}
\end{figure}

\subsection{Núcleos}
Os chips da série M empregam uma arquitetura heterogênea de múltiplos núcleos - uma abordagem inspirada na arquitetura ARM big.LITTLE - combinando núcleos de alto desempenho (Firestorm) e núcleos de alta eficiência (Icestorm). Essa combinação permite que o sistema ajuste seu consumo energético com base na tarefa em execução, dando preferência para núcleos de alto desempenho em tarefas intensivas, enquanto, para operações mais leves, núcleos de alta eficiência energética são usados.

\subsection{Memória}
Uma das inovações fundamentais dessa arquitetura é o conceito de memória unificada. Diferente dos designs tradicionais, nos quais CPU e GPU utilizam memórias separadas, a arquitetura dos chips M permite que todos os processadores acessem a mesma memória física. Isso reduz a necessidade de cópias redundantes de dados e melhora a comunicação entre CPU e GPU, resultando em um desempenho mais eficiente e menores tempos de resposta. Tal memória unificada também conta com uma largura de banda de 68.25 GB/s para o processador M1, podendo chegar até 400 GB/s para o M1 Max.

\hspace{+15pt}

\section{Desempenho}
\hspace{+15pt}


\section{Comparativo entre modelos}
\hspace{+15pt}


\section{Conclusão}
\hspace{+15pt}



\newpage
\begin{thebibliography}{00}
\bibitem{brain_like_computing} Ou, W., Xiao, S., Zhu, C., Han, W., \& Zhang, Q. (2022). \textbf{An overview of brain-like computing: Architecture, applications, and future trends}. Frontiers in Neurorobotics, 16, 1041108.

\bibitem{apple_silicon_overview} Brown, M., \& Williams, J. (2022). \textbf{Apple Silicon: Powering the Future of Computing}. IEEE Xplore. Disponível em: https://ieeexplore.ieee.org/abstract/document/9926315.

\bibitem{apple_at_work} Apple Inc. (2021). \textbf{Apple at Work – M1 Overview}. Apple Business. Disponível em: https://www.apple.com/br/business/mac/pdf/Apple-at-Work-M1-Overview.pdf.

\bibitem{hardware_acceleration} Chen, Y., \& Li, X. (2021). \textbf{Hardware Acceleration Techniques for Next-Generation Computing}. In S. Kumar \& P. Jones (Eds.), Emerging Technologies in Computing (pp. 509-523). Springer, Cham. Disponível em: https://link.springer.com/chapter/10.1007/978-3-030-93677-8\_48.
\end{thebibliography}
\end{document}
